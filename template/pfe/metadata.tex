\sethidden{0} % 1 : cache les attributs non renseignés (si pas souhaités) 0 : indique en rouges les attributs manquants, et leur position sur le document
\setimageleft{images/sogeti.png} % nom de l'image à gauche
\setimageright{images/Logo_ENSIMAG.png} % nom de l'image à droite

\setaddrentreprise{43 rue pré Gaudry} % rue
\setauteurlist{Guillaume \textsc{de Bigault de Granrut}} % auteurs (auteur \\ auteur...)
\setentreprise{Sogeti} % nom de l'entreprise
\setfilliere{3e année Ingénierie des Systèmes d'Information}
\setperiode{1 mars 2021 - 27 Août 2021 -- 6 mois} % dates + durée
\setrespostage{Alexandre \textsc{Suin}} % maitre de stage dans l'entreprise
\setschool{Grenoble-INP ENSIMAG} % nom de l'école
\setschoolbis{Ecole Nationale Supérieure d'Informatique et de Mathématiques Appliquées de Grenoble} % acronyme détaillé du nom de l'école
\setsoustitre{Rapport de Projet de Fin d'Etudes} % (rapport de pfe)
\settitre{Récupération de métriques}
\settuteurstage{Roland \textsc{Groz}} % tuteur au sein de l'école
\setvilleentreprise{69007 Lyon, France} % code postal + ville

\setbooltoc{1} % 1 si sommaire, 0 sinon
\setbooltof{1} % 1 si liste des figures 0 sinon
\setbooltot{1} % 1 si liste de tables 0 sinon
\setbooltol{1}