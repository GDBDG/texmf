% ------------------Options des couleurs
%pas de couleur (par défaut)
\definecolor{ColorSection}{rgb}{0,0,0}
\definecolor{ColorSubSection}{rgb}{0,0,0}
\definecolor{ColorTodo}{rgb}{1,0,0}
\definecolor{Description}{rgb}{0.5,0.5,0.5}
%
\renewcommand{\thesection}{\Alph{section}}

% option : colorised  (merci Tugdual pour les couleurs)
\DeclareOption{colorised}{
  \definecolor{ColorSection}{rgb}{.204,.353,.541}
  \definecolor{ColorSubSection}{rgb}{.31,.506,.741}
}


%------------------Options de mise en page
%* if no option
\titleformat{\section}
  {\bfseries\Large\color{ColorSection}}
  {\Alph{section}}
  {0.4em}
  {\underline}

\titleformat{\subsection}
  {\bfseries\large\color{ColorSubSection}}
  {\hspace{1em}\arabic{subsection}}
  {0.5em}
  {\underline}

\titleformat{\subsubsection}
  {\bfseries\color{ColorSubSection}}
  {\hspace{2.5em} \alph{subsubsection})}
  {0.5em}
  {\dashuline}

%* option : devie (style chargé, inspiré de mes cours de spé)
\DeclareOption{devie}{
  \renewcommand{\thesection}{\Roman{section}}
  \titleformat{\section}
    {\bfseries\Large\scshape\color{ColorSection}}
    {\fbox{\fbox{\Roman{section}.}} -}
    {0.4em}
    {\uuline}

  \titleformat{\subsection}
    {\bfseries\large\color{ColorSubSection}}
    {\hspace{1em} \fbox{\arabic{subsection}.}}
    {0.5em}
    {\underline}
}
% Fin option devie

%* option : sobre (style plus léger, avec chiffre romains pour les sections)
\DeclareOption{sobre}{
  \renewcommand{\thesection}{\Roman{section}}
  \titleformat{\section}
    {\bfseries\Large\scshape\color{ColorSection}}
    {\Roman{section}. -}
    {0.5em}
    {\underline}

  \titleformat{\subsection}
    {\bfseries\large\color{ColorSubSection}}
    {\hspace{1em} \arabic{subsection}.}
    {0.5em}
    {\underline}
  
}% Fin sobre
% Sections avec description
\newcommand{\setDescriptionSection}[1]{\renewcommand{\descriptionSection}{\textcolor{Description}{#1}}}
\newcommand{\descriptionSection}{}
\newcommandx{\sectionC}[3][2=,3=1]{\newpage \section{#1}}
\newcommand{\descriptionRapport}{}
\newcommand{\setDescriptionRapport}[1]{\renewcommand{\descriptionRapport}{#1}}
\newcommand{\showDescriptionRapport}{}
\DeclareOption{dev}{
    \renewcommand{\showDescriptionRapport}{\textcolor{Description}{\descriptionRapport}}
    \renewcommandx{\sectionC}[3][2=,3=1]{
        \IfEq{#3}{1}{
             \titleformat{\section}
                {\bfseries\Large\scshape\color{ColorTodo}}
                {\Roman{section}. - todo}
                {0.5em}
                {\underline}
        }{}
        \IfEq{#2}{}{
            \newpage
            \section{#1}
        }
        {
            \newpage
            \section{#1 --  [#2] pages}
        }
         \IfEq{#3}{1}{
             \titleformat{\section}
                {\bfseries\Large\scshape\color{ColorSection}}
                {\Roman{section}. -}
                {0.5em}
                {\underline}
        }{}
        \descriptionSection
        \setDescriptionSection{}
    }
}

\newcommand{\censure}[1]{#1}
\DeclareOption{censure}{
    \renewcommand{\censure}[1]{{\huge \textcolor{red}{Passage censuré, en attente de validation}}}
}
\DeclareOption{showscensure}{
    \renewcommand{\censure}[1]{\textcolor{red}{Passage qui doit être autorisé début #1 fin du passage qui doit être autorisé }}
}
%* Indispensable
\ProcessOptions\relax
\pgfplotsset{compat=1.16}