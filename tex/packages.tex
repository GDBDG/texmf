% !------------------------------ Les indispensables ------------------

\usepackage[french]{babel}
\usepackage[utf8]{inputenc}
\usepackage[T1]{fontenc}
\usepackage{amssymb,amsmath} %defini tous les symboles maths http://milde.users.sourceforge.net/LUCR/Math/mathpackages/amssymb-symbols.pdf
% !------------------------- Les miens -------------------------------
\usepackage{header} % Mine, utile pour page de garde et header (et rien d'autre)
\usepackage{utilitaire} % Mine, d&finit un certain nombre de primitive de base

%! ---------------------Coloration + décors

\usepackage{fancybox} % Encadrer du texte et autre https://ctan.crest.fr/tex-archive/macros/latex/contrib/fancybox/fancybox-doc.pdf
\usepackage[tables, svgnames]{xcolor} %!doit être chargé avant les packegs qui import xcolor (ex : pgfgantt) % Donne des couleurs par défaut 


%! --------------- Entete et mise en page------------------------------
\usepackage{titling} % Permet d'accéder à /thetitle, author et date, http://www.ctex.org/documents/packages/layout/titling.pdf
\usepackage{lastpage}% Pour avoir le nombre de page dans le header https://www.ctan.org/pkg/lastpage
\usepackage{layout} % donne un gabarit des marges http://web.mit.edu/texsrc/source/latex/layouts/layman.pdf (dispensable pour un doc lambda)
\usepackage{fancyhdr} % Entete et pied de page https://aurelienpierre.com/wp-content/uploads/fancyhdr_fr.pdf
\usepackage{geometry} % permet de configurer les marges http://texdoc.net/texmf-dist/doc/latex/geometry/geometry.pdf
\usepackage{setspace} % permet de changer l'interligne (\begin{onehalfspace} ou {doublespace}) https://ctan.org/tex-archive/macros/latex/contrib/setspace
\usepackage{xspace} % Permet de mettre des espaces correctement dans des nouvelles commandes avec \xspace
\usepackage{titlesec} % Permet de bidouiller les titres https://www.ctan.org/pkg/titlesec
\usepackage{ulem}% Pour les soulignages https://ctan.org/pkg/ulem
\usepackage{lipsum}% https://ctan.org/pkg/lipsum
\usepackage{tocloft} %http://mirrors.ibiblio.org/CTAN/macros/latex/contrib/tocloft/tocloft.pdf pour bidouiller le sommaire
\usepackage{ifmtarg} % pour gérer les arguments vides de commande https://ctan.org/pkg/ifmtarg
\usepackage{biblatex}
\usepackage{pgfgantt}
\usepackage[babel=true, kerning=true]{microtype}  %https://tex.stackexchange.com/questions/289852/package-tikz-error-with-pgfgantt (erreurs pgfgantt et babel french, permet de fix
\usepackage[nottoc, notlof, notlot]{tocbibind}

%!-------------------------- Ajout d'images 

\usepackage{graphicx} % Pour ajouter des images
%\usepackage{slashbox} % permet d'utiliser \backslashbox{Texte dessous}{Texte dessus} pour les tableaux (TO INSTALL)

% !---------------------------- Fichier sources code

\usepackage{listings} %https://mirror.ibcp.fr/pub/CTAN/macros/latex/contrib/listings/listings.pdf
\usepackage{listingsutf8} % Permet d'utiliser listings en utf8

% !----------------------------- Les tableaux ------------------------%

\usepackage{array} % Pour array et tabular https://www.ctan.org/pkg/array
\usepackage{tkz-tab} %  Permet de faire des tableaux avec tikz http://ctan.math.washington.edu/tex-archive/macros/latex/contrib/tkz/tkz-tab/doc/tkz-tab-screen.pdf 
\usepackage{multirow} %  Permet de fusionner des lignes de tableau http://ctan.mines-albi.fr/macros/latex/contrib/multirow/multirow.pdf
\usepackage{multicol} % Permet de fusionner des colonnes de tableau http://mirrors.ircam.fr/pub/CTAN/macros/latex/required/tools/multicol.pdf
\usepackage{hhline} % Permet de faire des lignes propres dans des tableaux https://ctan.crest.fr/tex-archive/macros/latex/required/tools/hhline.pdf
\usepackage{arydshln} % permet les lignes pointillé dans les tableaux
\usepackage[]{blkarray} % je sais plus pourquoi je m'en sers  https://www.ctan.org/pkg/blkarray
%!----------------------------- Les maths de bases

\usepackage{amsthm} % Utilisé pour les théorèmes http://mirrors.standaloneinstaller.com/ctan/macros/latex/required/amscls/doc/amsthdoc.pdf (doit être appelé après amsmath)
%\usepackage{amsfonts} % Pour indicatrice https://www.ctan.org/pkg/amsfonts (pas utilisé, à étudier)
\usepackage{systeme} % Permet de faire des equations simplement http://ctan.tetaneutral.net/macros/generic/systeme/systeme_fr.pdf
\usepackage{xstring} % Truc obscur de macros http://mirror.ibcp.fr/pub/CTAN/macros/generic/xstring/xstring-fr.pdf %*Utilisé par systeme
\usepackage{stmaryrd} % ajoute des symboles, dont les brackets (doubles crochets) https://www.ctan.org/pkg/stmaryrd
\usepackage{verbatim} % Réimplente verbatim, http://distrib-coffee.ipsl.jussieu.fr/pub/mirrors/ctan/macros/latex/required/tools/verbatim.pdf %? Sert à quoi exactement ? 
\usepackage{verbatimbox} % http://distrib-coffee.ipsl.jussieu.fr/pub/mirrors/ctan/macros/latex/contrib/verbatimbox/verbatimbox.pdf %? Sert à quoi ????
\usepackage{etoolbox} %http://mirrors.ircam.fr/pub/CTAN/macros/latex/contrib/etoolbox/etoolbox.pdf Obscur, pour definir commandes %? Se renseigner

%! ---------------------------Figures ----------------------------

%\usepackage{subfigure} %! En conflit, mais pas sur overleaf, à vérifier
\usepackage{subcaption} % Pour les figures https://ctan.crest.fr/tex-archive/macros/latex/contrib/caption/subcaption.pdf %? Se renseigner
\usepackage{float} % Sert à la gestion des objets flottants (figures ...), implémente le [H] %https://ctan.crest.fr/tex-archive/macros/latex/contrib/float/float.pdf
\usepackage{pgf,tikz, pgfplots} %http://cremeronline.com/LaTeX/minimaltikz.pdf (une doc minimale)
\usetikzlibrary{automata, positioning, arrows} % Pour tracer des automates https://ctan.org/pkg/automata https://notendur.hi.is/aee11/automataLatexGen/
\usetikzlibrary{matrix}
\usetikzlibrary{calc}
\usetikzlibrary{fit}
\usetikzlibrary{arrows}

%!---------------------------- Utiles pour le latex
\usepackage{xargs} % Facilite la vie pour les arguments par défaut https://ctan.org/pkg/xargs

%!------------------------------Autres------------------
\usepackage{eurosym}% permet de définir le symbole euro

%!---------------------------------hyperref (à mettre en dernier)
\usepackage{hyperref} % Met des. liens dans un sommaire, doit être le DERNIER DES PACKAGES, compiler 2 fois https://ctan.org/pkg/hyperref
\hypersetup{
    colorlinks=true,
    linkcolor=blue,
    filecolor=magenta,
    urlcolor=cyan,
    citecolor=magenta,
}
\urlstyle{same}

\usepackage[toc, automake, acronym]{glossaries}
